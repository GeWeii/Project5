
\section{Introduction}

Dans le marché financier, tout le monde cherche à gérer et optimiser son portefeuille pour obtenir le rendement attendu. Il y a une trentaine d'années, beaucoup de monde a commencé à essayer de prédire le rendement de l'action en utilisant le volume et les prix historiques, le prix le plus haut, le plus bas, d'ouverture et de clôture. Plus tard, des indicateurs techniques (TIs), qui représentent la tendance, la volatilité et l'ampleur du marché, ont été développés et calculés à partir de volume et de prix pour rendre la prédiction plus efficace.\\

Dans le cadre de notre projet, nous travaillons sur les données historiques de TI des actions dans l'indice CAC40 pour prédire le rendement en appliquant un modèle de réseaux de neurones. Il y a trois facteurs principaux qui vont impacter la prédiction du rendement, l'horizon, la longueur des données hisroriques utilisée et le nombre de TIs.\\

Dans la vraie vie, les investisseurs souvent font des opérations (par exemple, achat ou vent d'une action) pour manipuler leur portefeuille et laissent passer le temps sans prendre aucune mesure. Alors que cela peut être un bon choix lors de l'achat, et peut devenir de moins en monis performant au fils du temps. Par conséquent, il est très important de décider l'horizon approprié entre les deux opérations consécutives en prenant en compte le risque de la volatilité du marché, sachant qu'il est coûteux de faire des opérations fréquemment.\\

Comme nous travaillons sur des séries temporelles, qui d'ailleurs ne sont pas stationnaires de longue mémoire, il est intéressant de pouvoir déterminer la durée de données d'apprentissage pour mieux prédire le rendement à un horizon prédéfini, ainsi que le nombre de TIs.\\

Ce rapport est pour vous présenter notre projet en détail, qui contient 5 parties en global. D’abord, nous expliquons le context du projet, nous montrons notre problématique et les données fournies par nos encadrants. Ensuite, nous présentons le modèle de réseaux neurones pour réaliser la prédiction, qui consiste en la construction de la base d’apprentissage et les paramètres du modèle. Nous analysons les résultats obtenus et aussi évaluons notre modèle par les indicateurs statistiques. Dans la partie suivante, nous donnons des points à améliorer pour la continuation de ce projet. Ce rapport se termine par une conclusion synthétique. 


\subsection{Problématique}

Un portefeuille d'actifs est construit par plein d'actions des poids différents, les actions sont impactées par plusieurs facteurs comme le secteur, le pays, etc. Pour gérer un portefeuille d'actifs, l'objectif est de maximiser le rendement global, c'est-à-dire qu'il faut être capable de faire la décision d'acheter ou de vendre des actions pour optimiser le rendement de chaque action. Dans le cadre de notre projet, nous avons besoin d’optimiser d'un portefeuille CAC40. Comme les séries temporelles sont non stationnaires, donc il est très difficile de faire la prédiction pour chaque période.

\subsection{Présentation de données}

Le portefeuille CAC40 est construit par 49 actions françaises et chaque action a ses comportements complètement différents. Nous avons un fichier qui contient le prix (ouverture, clôture, haut, bas) et le volume de chaque action pendant presque 15 ans (2000-2015). Les données dans ce fichier seront utilisées pour calculer le rendement d'un horizon déterminé. Pour une série temporelle, si nous prenons une période trop courte, nous n'aurons pas assez d'informations pour faire la prédiction. En revanche, elle n'a pas une forte corrélation pendant une longue période, les données très anciennes ne sont pas utile de prédire la situation actuelle.\\

Nous avons aussi un fichier ayant les 13 « technical indicators » calculés à partir de ces séries temporelles. Les TIs se situent dans 4 catégories : le volume, la volatilité, la tendance et le momentum.
