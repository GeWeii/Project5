\section{Difficultés rencontrées et points à améliorer}

\paragraph{Difficultés rencontrées}
Les marchés financiers ne sont pas du tout stables, ni prévisibles, et même un message de Twitter peut les influencer. Par conséquent, il est difficile de prévoir le rendement d'une action sur le marché. De plus, la performance d'une action varie beaucoup, non seulement en fonction de facteurs économiques tels que le taux d'intérêt et l'inflation, mais aussi en fonction du secteur où elle se trouve, la taille de l'entreprise, les activités des entreprises concurrentes, et cætera.

Nous travaillons sur les actions européennes dans l'indice CAC40 qui est composé de 49 actions de différents secteurs : banque et assurance, transports, technologie, études et conseil, et cætera. Appliquer notre modèle de prédiction à toutes ces actions et comparer les résultats des autres actions du même secteur représente donc un projet conséquent. De plus, l'outil que nous utilisons pour l'implémentation est iPython notebook. Son avantage est qu'il est aisé d'y réaliser des tests et de visualiser les résultats, mais il n'est pas très efficace pour lancer de nombreux tests de différents scénarios et pour vérifier la performance du modèle.\\

\paragraph{Points à améliorer}
Nous avons réussi à réaliser tous les tests sur l'action de Renault, de PSA et d'Alcatel-Lucent. Les travaux futurs seront d'améliorer la performace de notre modèle pour minimiser l'erreur de la prédiction, par exemple, d'implémenter les autres scénarios potentiels pour trouver une meilleure combinaison de paramètres pour le modèle. Il est aussi intéressant d'appliquer notre modèle de prédiction à toutes les autres actions pour voir si sa capacité de prédiction est générique.

L'objectif final de notre modèle est de permettre aux traders de prendre des décisions à un horizon donné, d'acheter, de vendre ou de ne rien faire sur les actions dans leur portefeuille afin d'optimiser ce dernier. C'est un travail à réaliser à partir du modèle actuel. Par ailleurs, nous proposons d'implémenter l'algorithme en C++ pour qu'il soit plus déterministe et efficace une fois le modèle amélioré.