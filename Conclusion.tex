
\section{Conclusion}

Ce projet constitue une belle opportunité pour découvrir la nouvelle technique pour prédire la série temporelle, cela nous permettra d’acquérir des connaissances sur le réseau de neurones et les événements financiers. Il s’est déroulé du 23 octobre 2017 au 14 mars 2018, nous avons commencé par la compréhsion du sujet et les recherches, suivie par une phase de conception, nous avons discuté beaucoup sur les paramètres de la base, puisque cela est lié significativement à la performance du modèle. Le plus important est la phase de test, nous avons commencé à faire des analyses et les relier avec les événements dans la vraie vie. Cependant, nous avons eu beaucoup de difficultés dans cette période, parce que le temps d'exécution est trop long, ainsi les résultats sont très variés. Nous n'avons pas réussi à faire tous les tests sur toutes les actions, par contre nous avons déjà obtenu un premier résultat global sur l'action Renault. Nous pensons que pour aller plus loin, il faut continuer à tester les scénarios sur tout le portefeuille de CAC40. \\

Au point de vue personnelle, nous avons pratiqué beaucoup d'analyse et de la prédiction de données sous ipython, c'est utile pour la vie professionnelle dans le futur. Même s'il est très difficile d'avoir un résultat idéal, nous avons déjà appris plein de démarches pour déployer ce genre de sujet. Il vaut mieux avoir plus de temps pour finir tous les tests, ainsi que les comparaisons entre les différents cas. 